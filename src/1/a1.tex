\documentclass{article}
\title{Kenapa milih RSC?}
\author{Haikal}
\date{4 Februari 2025}
\begin{document}
Sebenarnya ketertarikan saya ke \\textit{UAV} dimulai saat SMA, OPSI atau Olimpiade \\ Penelitian Siswa Indonesia dimana siswa-siswa seluruh Indonesia berkompetisi dalam \\bidang penelitian. Saya dan rekan awalnya bingung akan mengambil judul apa, \\mengingat passion kami adalah berinovasi maka saya dan rekan saya memutuskan untuk \\mengabil judul “Rancang Bangun UAV Pendeteksi Kedalaman “. Jujur, saat mengambil \\judul itu, kami bahkan tidak tahu bagaimana robot kami akan beroperasi. Hanya \\gambaran kasar terkait hal-hal seperti bagaimana bentukan dan sistem propulsi kapal. \\Kami memilih langsung membeli \\textit{Rasbery pi} dan \\textit{Arduino UNO} dengan \\servo dan sensor \\textit{ultrasonic}. Idenya adalah sensor ultrasonic tersebut akan \\digunakan untuk mendeteksi kedalam Sungai. Kami melihat \\textit{CNN(Convoluted \\Neural Network} memiliki potensi cukup bagus untuk menebak dan mengirimkan \\instruksi terkait posisi sampah. 

Kami pun mencari mentor dari tim robotik sekolah. Kami berkenalan dengan kak Kusuma, \\Mahasiswa elektro ITS alumni sekolah kami. Di titik ini, saya masih kesulitan dengan \\pekerjaan \\textit{nguli} seperti \\textit{sudo apt} dan bagaimana menginstall OS. Kak \\Kusuma mengatakan “Pake YoLO enak udah aku coba, kamu bakat masuk aja robotik”. \\Namun disitu kami masih berpendapat bahwa CNN memiliki potensi mengingat \\keterbatasan kekuatan komputasi. Saya menemukan model  TFLITE sebagai kandidat \\utama. 

Kami menggunakan GCS (Google Cloud Server) sebagai komputer training model dan \\inference. Sebenarnya inference awalmya akan dilakukan di raspberry pi namun saat \\saya coba menggunakan model dan kode yang ada di repo tflite, fps yang didapat tidak \\maksimal (4 fps). Disini saya sadar harusnya membeli raspberry pi 5 dan bukan 4 hanya \\dengan perbedaan harga tak signifikan. Metode inference adalah mengirimkan foto \\dalam nilai RGB melalui HTTP menuju server untuk inference. Karena server memiliki \\RAM massif dan Prosesor kencang, Kami menggunakan Multithreading untuk \\mempercepart inference(dengan bantuan AI).

Bukan hanya masalah AI dan OpenCV yang rumit melainkan jaringan atau koneksi \\raspberry pi yang selalu gagal mengirimkan request ke server. Hingga saat ini saya belum \\paham apa yang menjadi masalahnya. TURN server hingga UDP port GCS yang dibuka \\atau digunakan. Suatu saat, semuanya bekerja.  Bukan hanya software, kami juga sempat \\bingung tentang material kapal. Resin menjadi solusi saat kolega ayah saya \\merekomendasikan setelah saya menjelaskan proyek ini padanya. “pake resin aja mas \\pasti ngapung” begitu katanya. Memang rumit namun Ketika menemukan sebuah solusi, \\saya merasa telah berhasil melakukan sesuatu. Meskipun karena keterbatasan dana dan \\waktu proyek tidak selsai, saya berkembang jauh dari awalnya tidak tahu apa-apa tentang \\AI hingga bisa melatih pretrained model sendiri dan mengintegrasikannya ke dalam \\sistem robotic melalui serial. Perjalanan ini menunjukkan jika anda gigih dan memang \\mau belajar maka anda akan bisa melakukannya. 

Aksantara menurut saya membuat saya meneruskan ketertarikan terhadap robotik. RCS \\adalah departemen yang paling cocok dimana saya bisa belajar bagaimana caranya \\membuat robot-robot canggih dan keren. Ditambah lagi robot itu bisa terbang. 

\end{document}
